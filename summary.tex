%\makesummary
%Generic YSO waffle
\summary
{
\def\baselinestretch{1.7}
In the previous fifty years it has become clear that jets and outflows play a vital role in the formation of stars and compact objects. Jets from young stellar objects typically show Herbig-Haro knots and bow shocks.
Additionally, it now appears that (1) most stars form in binaries, and (2) jets from young stars are multiple and episodic outflows.
Several groups have carried out large-scale simulations of jets, but often assuming a uniform ambient medium and a single disk and star.
In this thesis the problems associated with non-uniform media and binary systems are explored.
% Scientific questions
In order to understand the role of jets in star formation the questions are asked: how do jets from binary stars behave? What is the effect of the prehistory of jets on their collimation, acceleration and morphology?
% Tools
\\To answer these questions, a parallel adaptive-grid magnetohydrodynamics code, ATLAS, is modified to include optically thin atomic radiative cooling losses.
The code is rigorously tested, with particular reference to the shock-capturing and the radiative cooling.
The tests used include one-dimensional shock-tube tests, two-dimensional blast waves, double Mach reflection of a strong shock from a wedge, the overstable radiatively cooling shock, and the Orszag-Tang vortex. 
A comparison of the code with another code, PLUTO, for the type of jet problems
solved in this thesis is also performed.
% Results
Using ATLAS, the propagation of jets in complex environments is studied.
The first ever simulations of binary jets are performed.
Three aspects of the problem are studied, the effects of source orbiting, the effects of interaction, and the role of the magnetic field.
It is shown that jets from binary stars can interact and the signature of the interaction is demonstrated.
The negligible effect of source orbiting is demonstrated.
A toroidal magnetic field is placed in the ambient environment and further accentuates the interaction.
Following on from this work, the evolution of the jet when the environment is not uniform is studied.
Simulations have been performed which track the evolution of a jet in an partially evacuated cavity.
The parameter space of the problem is explored in axisymmetry.
The strong effect of the cavity on the recollimation, the acceleration and the radiative cooling losses is demonstrated.
%This gives a great deal of information about the large-scale structure of the molecular cloud complexes which are the nurseries of star formation.
%Outflows and jets from young stellar objects entering the interstellar medium at supersonic speeds produce shocks from which emission is detected.
%Simulations are a powerful tool to study jets, however shocks present stability problems.
%model their effects using a 3D atomic cooling magnetohydrodynamics code ATLAS. ATLAS makes use of Adaptive Mesh Refinement and the Piecewise Parabolic Method for time-advection, both of which are uniquely suited to shock simulations.
%Testing of the code has focussed on its ability to model shocks.
%present results from new models of jets from binary protostars and a jet in an evacuated cavity emerging from a Class I young stellar object.
%show that the effect of the binarity of the source is to disrupt the jet and cause it to lose the protection of its bow shock.
%model a single jet propagating through an evacuated cavity and show its recollimation.
}
