\chapter{Derivation of the energy equation from the second moment of the Boltzmann equation}
\label{MHDDerivation}
This derivation uses parts of \citet{2006Kominsky}.
The collisionless form of the Boltzmann equation is:
\begin{equation}
\frac{\partial{ n \langle \chi \rangle }}{\partial{t}}
+
 {\boldsymbol{\nabla}}_x n{\langle \mathbf{v} \chi \rangle}
+
n
\langle
\mathbf{a}
{\boldsymbol{\nabla}}_v . { \chi }
\rangle
=0
\end{equation}

where $\chi$ represents a conserved quantity such as density, total energy or momentum.




Taking the second moment,  $\chi  = \frac{1}{2} m v^2 =  \frac{1}{2} m  \mathbf{v \cdot v}$, gives
\begin{equation}
\frac{\partial}{\partial t}\left[
\rho
\left(
\mathbf {u \cdot u }
+
\langle
\mathbf {w \cdot w }
\rangle
\right)
\right]
+ {\boldsymbol{\nabla}}_x
\cdot
\left(
\frac{1}{2}
\rho
\langle
\left(
\mathbf{v}
\cdot
\mathbf{v}
\right)
\mathbf{v}
\rangle
\right)
+ \rho \mathbf{a} . {\boldsymbol{\nabla}}_v \frac{1}{2} \mathbf{u}^2
=0
\label{SecondMoment}
\end{equation}


Taking the definition of $ P = \frac{1}{d }\langle \mathbf {w \cdot w } \rangle $ 
where $d$ is the dimensionality of the system (often 3). Introducing the adiabatic index or ratio of specific heats;
\begin{equation}
\gamma = \frac{d+2}{d}
\end{equation}
for a monatomic gas.
\begin{equation}
P = \frac{1}{d} \rho \langle w^2 \rangle = \frac{\gamma -1}{\gamma} \rho \langle w^2 \rangle
\end{equation}


Define the energy density, 
\begin{equation}
E = \frac{1}{2} \rho u^2 + \frac{p}{\gamma -1}
\end{equation}


The first term in Equation \ref{SecondMoment} is then
\begin{equation}
\frac{\partial}{\partial t}\left[
\rho
\left(
\mathbf {u \cdot u }
+
\langle
\mathbf {w \cdot w }
\rangle
\right)
\right]
= 
\frac{\partial E }{\partial t}
\end{equation}

The second term:


\begin{eqnarray*}
\boldsymbol{\nabla} .\left( \frac{1}{2} \rho \langle \mathbf{\left( v \cdot v\right) v}\rangle \right) &=&     
\boldsymbol{\nabla} .\left( \frac{1}{2} \rho \langle \mathbf{\left(  \left( \mathbf {u+w}\right)\cdot  \left( \mathbf {u+w}\right)\right) \left( \mathbf {u+w}\right) }\rangle \right)  
\\
 &=& 
\boldsymbol{\nabla} .\left( \frac{1}{2} \rho \langle \mathbf{\left( \mathbf {u^\mathrm{2}+\mathrm{2}u \cdot w+w^\mathrm{2}}\right) \left( \mathbf {u+w}\right) }\rangle \right)  
\\
 &=& 
\boldsymbol{\nabla} .\left( 
\frac{1}{2} \rho   u^2 \mathbf{u} 
+\frac{1}{2} \rho \langle w^2 \rangle \mathbf{u} 
+ \rho u w^2 
+\frac{1}{2} \rho  w^3 
\right)  
\\
 &=&
\boldsymbol{\nabla} . \left(
E \mathbf{u}
+
P \mathbf{u}
+ \frac{1}{2} \rho  w^3
\right)
\\
\end{eqnarray*}

Excluding gravity, the acceleration due to electromagnetic forces is:
\begin{equation}
\mathbf{a}= \frac{q}{m} \left( \mathbf {E + u \times B}\right)
\end{equation}

The acceleration term is then:
\begin{equation}
{\boldsymbol{\nabla}}_v \chi 
= 
\frac{1}{2} m {\boldsymbol{\nabla}}_v \left( \mathbf{v \cdot v} \right) 
=
\mathbf{v} \cdot m {\boldsymbol{\nabla}}_v \left( \mathbf{ v} \right) 
=
m \mathbf{v}
\end{equation}
\begin{equation}
- n \langle \mathbf{a} \cdot {\boldsymbol{\nabla}}_v  { \chi } \rangle 
=
- n \langle \frac{\mathbf{F}}{m} \cdot m \mathbf{v} \rangle 
=
- n \langle \mathbf{F} \cdot \mathbf{v} \rangle 
\end{equation}

Now,
\begin{equation}
 \langle \mathbf{F} \cdot \mathbf{v} \rangle 
=
 \langle \mathbf{F} \cdot \mathbf{u} \rangle
+
 \langle \mathbf{F} \cdot \mathbf{w} \rangle
=
 \langle \mathbf{F} \rangle \mathbf{u} 
+
 \langle \mathbf{F} \cdot \mathbf{w} \rangle
\end{equation}

For all velocity-independent forces $\mathbf{F}$
\begin{equation}
\langle \mathbf{F \cdot w} \rangle
=
\mathbf{F \cdot} \langle \mathbf{ w} \rangle
=0
\end{equation}

For the Lorentz force, 
\begin{eqnarray*}
\langle \mathbf{F \cdot w} \rangle
&=&
\langle
\left(
q
\left(
\mathbf{E + v \times B}
\right)
\right)
\cdot
\mathbf{w}
\rangle\\
&=&
q \langle \mathbf{\left( v \times B \right) \cdot  w} \rangle
+
q \langle \mathbf{E \cdot w} \rangle
\\
&=&
q \mathbf{u \times B} \langle w \rangle
+
q \langle \left(  w \times B  \right)  \cdot w \rangle
\\
\end{eqnarray*}


The moment equation is now:

\begin{equation}
\frac{\partial}{\partial t}E
+
{\boldsymbol{\nabla}}_x
\left(
u \left( E +P \right)
\right)
-
nq\mathbf{u \cdot E}
=0
\end{equation}

Defining the current density:
\begin{equation}
\mathbf{J} = n q \mathbf{u}
\end{equation}


\begin{equation}
\frac{\partial}{\partial t}E
+
{\boldsymbol{\nabla}}_x
\left(
u \left( E +P \right)
\right)
-
\mathbf{J \cdot E}
=0
\end{equation}

Using vector identities, and assuming infinite conductivity,  $\frac{\mathbf J }{\sigma}= \mathbf {E + u \times B} =0$ the $ \mathbf{J \cdot E} $ term can be written as:

\begin{eqnarray*}
 \mathbf{J \cdot E}
&=& 
  \mathbf{ \left( \boldsymbol{\nabla} \times B \right) \cdot E}
 \\
&=& 
  \mathbf{ \left( \boldsymbol{\nabla} \times E \right) \cdot B} -  \boldsymbol{\nabla} \cdot \left( \mathbf{E \times B} \right)
 \\
&=& 
   \frac{\partial \mathbf{B} }{\partial t} \cdot \mathbf{B} -  \boldsymbol{\nabla} \cdot \left( \mathbf{ u \times B  \times B} \right)
 \\
&=& 
  \frac{1}{2} \frac{\partial B^2 }{\partial t}  -  \boldsymbol{\nabla} \cdot \left( \mathbf{ \left( u \cdot B \right)  B } - B^2 \mathbf{u} \right)
 \\
\end{eqnarray*}

 
Finally, the conservation of energy is:

\begin{equation}
\frac{\partial E}{\partial t}
+
{\boldsymbol{\nabla}}_x
\left(
u \left( E +P + \frac{1}{2} B^2 \right)
- \mathbf{ \left( u \cdot B \right)  B }
\right)
=0
\end{equation}

